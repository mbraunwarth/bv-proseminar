%% Template file for seminar papers
%% based on bare_conf.tex V 1.3 (2007/01/11) by Michael Shell (http://www.michaelshell.org/)
% *** Do not adjust lengths that control margins, column widths, etc. ***
% *** Do not use packages that alter fonts (such as pslatex).         ***

\documentclass[conference]{IEEEtran}


\usepackage[utf8]{inputenc} % Kodierung
\usepackage[ngerman]{babel} % Sprache
\usepackage{makeidx}  % allows for indexgeneration

\usepackage{math}
\usepackage{graphicx}
\usepackage{url}

% correct bad hyphenation here
%\hyphenation{op-tical net-works semi-conduc-tor}


\begin{document}
%
% paper title
% can use linebreaks \\ within to get better formatting as desired
\title{Bilaterale Filter und Diffusionsfilter}


\author{\IEEEauthorblockN{Markus Braunwarth}
\IEEEauthorblockA{
E-Mail: {\tt mbraunwarth@uni-koblenz.de}, Matrikelnummer: 215200130 \\
Proseminar Bildverarbeitung, Wintersemester 2018/19, Universität Koblenz-Landau
}}


% make the title area
\maketitle


\begin{abstract}

% erstsen Satz ergänzen (zu knapp)
Im Folgenden sollen Bilaterale Filter und die anisotropische Diffusion betrachtet werden. Der Schwerpunkt liegt hierbei auf Methoden zur Glättung von Grauwert- und Farbbildern wobei Kanten nicht verschmiert werden sollen, wie es bei vielen Filtern der Fall ist. Dazu werden zuerst Glättungsfilter im allgemeinen betrachtet und die Probleme, welche Kanten beim glätten von Bildern verursachen können. 

\end{abstract}

% For peerreview papers, this IEEEtran command inserts a page break and
% creates the second title. It will be ignored for other modes.
\IEEEpeerreviewmaketitle



\section{Einleitung}

In einer Vielzahl von Anwendungen der Datenverarbeitung ist es oft wichtig, Rauschen in Bildern zu verringern ohne den Umriss von Objekten zu verschmieren. Unter der Verwendung Bilateraler und Diffusionsfilter lassen sich Bilder glätten ohne diese Kanten zu verschmieren.

Diese Arbeit erläutert zuerst die Unterschiede der linearen und nicht linearen Filter sowie jene zwischen domain und range Filtern. Es folgt ein Vergleich der Kantenerhaltenden Glättung zu Glättungsfiltern, welche Kanten verschmieren. Darüber hinaus wird die Klasse der Bilateralen Filter und die anisotropische Diffusion näher beleuchtet. Ergenzend werden die Filtermethoden sowohl in Grauwert- als auch in Farbbildern betrachtet.

% Dies ist das erste Kapitel, das allerdings nicht so gut ist wie in \cite{adams1979hitchhiker}.



\section{Glättungsfilter}

\section{Kantenerhaltende Filter}

\subsection{Bilaterale Filter}

\begin{equation}
\bm{I}'(\bm u) = \frac
{\sum_{i=-S}^{+S} \sum_{j=-S}^{+S} \bm I(u_1 + i, u_2 + j) w}
{\sum_{i=-S}^{+S} \sum_{j=-S}^{+S} w}
\end{equation}

mit den Gewichten

\begin{equation}
w(\bm \xi, \bm u) = 
	\exp \left( 
		\frac{-(\bm \xi - \bm u)^2} {2 \sigma_D^2} 
	\right)
	\exp \left( 
		\frac{-(\bm I (\bm \xi) - \bm I (\bm u))^2} {2 \sigma_R^2} 
	\right)
\end{equation}

\subsection{Anisotropische Diffusion}

\begin{equation}
\bm I^{(n)}(\bm u) = 
	\bm I^{(n-1)}(\bm u) + \alpha 
	\cdot \sum_{i=0}^3 g(|\delta_i (\bm I^{(n-1)}, \bm u)|) 
	\cdot \delta_i (\bm I^{(n-1)}, \bm u)
\end{equation}

\section{Von Graustufen zur Farbe}

\section{Fazit}

% Bib aktualisieren: F6 -> F6 -> F11 -> F6
\bibliographystyle{alpha}
\bibliography{literatur}


% that's all folks
\end{document}
